\documentclass[pageno]{jpaper}

\usepackage[normalem]{ulem}
\usepackage[]{color}
\usepackage{amssymb}
\usepackage{mathtools}

\newcommand{\horizbar}{\rule{\linewidth}{.5mm}}
\newcommand{\app}[1]{{\sc #1}}
 
\renewcommand{\em}{\it}
 
%\newcommand{\pray}{\texttt{p-ray}}
%\newcommand{\prays}{\texttt{p-ray}\ }

\newcommand{\coderecon}{\textit{Energy Scout}}

\newcommand{\BigO}[1]{${\cal O}(#1)$}
\newcommand{\BigOmega}[1]{$\Omega(#1)$}
\newcommand{\BigTheta}[1]{$\Theta(#1)$}

\newcommand{\figtitle}[1]{\textit{\textbf{#1\hspace{1mm}}}}

\newcommand{\ceiling}[1]{\left\lceil #1 \right\rceil}
\newcommand{\faM}{\lfloor \alpha M \rfloor}
%\newcommand{\C}[2]{{#1 \choose #2}}
 
%\newcommand{\comment}[1]{}
\newcommand{\ignore}[1]{}


%\newcommand{\boldparagraph}[1]{\vspace*{-2ex}\paragraph{#1}}
\newcommand{\boldparagraph}[1]{\vspace*{-0ex}\paragraph{#1}}

%%%%% SINGLE FIGURE
\def\cfigure[#1,#2,#3]{
\begin{figure}
\vspace*{0mm}
\begin{center}

\includegraphics[width=3in]{#1} 
 
\vspace*{0mm}\caption[]{#2
} \label{#3}
 
\vspace*{-5mm}
\end{center}
%\horizbar
\vspace*{-2mm}
\end{figure}}

%%%%% SINGLE FIGURE
\def\cfiguretemp[#1,#2,#3]{
\begin{figure}
\vspace*{0mm}
\begin{center}

\includegraphics[width=3.5in]{#1} 
 
\vspace*{-3mm}\caption[]{#2
} \label{#3}
 
\vspace*{-5mm}
\end{center}
%\horizbar
\vspace*{-2mm}
\end{figure}}

%%%%% SINGLE WIDE FIGURE
\def\wfigure[#1,#2,#3]{
\begin{figure*}
\vspace*{0mm}
\begin{center}
 
\includegraphics[width=6in]{#1} 
 
\vspace*{-3mm}\caption[]{#2
} \label{#3}
 
\vspace*{-5mm}
\end{center}
%\horizbar
\vspace*{-2mm}
\end{figure*}}

\newlength{\myVSpace}% the height of the box
\setlength{\myVSpace}{3ex}% the default, 
\newcommand\tT{\raisebox{-0\myVSpace}% asymmetric behaviour, normally .5
  {\rule{0pt}{\myVSpace}}%
}
\newcommand\tB{\raisebox{-.5\myVSpace}% asymmetric behaviour, normally .5
  {\rule{0pt}{\myVSpace}}%
}



%%%%% 3 FIGURES IN A ROW
\def\threefigurembt[#1,#2,#3,#4,#5,#6]{
\begin{figure*}
\vspace*{0mm}
\begin{center}

\begin{tabular}{cccc}
\includegraphics[width=1.25in, trim=50 0 375 125]{#1} & \includegraphics[width=1.5in, trim=50 140 350 200]{#2} &  \includegraphics[width=1.75in, trim=50 30 0 200]{#3} &  \includegraphics[width=1.75in, trim=80 120 0 0]{#4} \\
(a) & (b) & (c) &(d) \\
\end{tabular}

%%%%\vspace*{-3mm}
\caption[]{#5
} \label{#6}

%%%%\vspace*{-5mm}
\end{center}
%\horizbar
%%%%\vspace*{-2mm}
\end{figure*}}





%%%%% 3 FIGURES IN A ROW
\def\threefigure[#1,#2,#3,#4,#5]{
\begin{figure*}
\vspace*{0mm}
\begin{center}

\begin{tabular}{ccc}
\includegraphics[width=2in]{#1} & \includegraphics[width=2in]{#2} &  \includegraphics[width=2in]{#3} \\
(a) & (b) & (c) \\
\end{tabular}

\vspace*{-3mm}\caption[]{#4
} \label{#5}

\vspace*{-5mm}
\end{center}
%\horizbar
\vspace*{-2mm}
\end{figure*}}

%%%%%% DOUBLE FIGURE
\def\dcfigure[#1,#2,#3,#4,#5,#6]{
{
\begin{figure*}
\vspace*{0in}\
\begin{center}
\begin{minipage}[c]{3.25in}{
\includegraphics[width=3.25in]{#1} 
\vspace*{-3mm}\caption[]{#2} \label{#3} \
}\end{minipage}\hspace*{0.5in}\
\begin{minipage}[c]{3.25in}{
\includegraphics[width=3.25in]{#4} 
\vspace*{-3mm}\caption[]{#5}\label{#6} \
}\end{minipage}
\end{center}
\vspace*{0in}\
\end{figure*}
}
}

\def\dssfigure[#1,#2,#3,#4,#5,#6]{
{
\begin{figure*}
\vspace*{0.2in}\
\begin{center}
\begin{minipage}[c]{4in}{
\includegraphics[width=4in]{#1}
\vspace*{-3mm}\caption[]{#2} \label{#3} \
}\end{minipage}\hspace*{0.5in}\
\begin{minipage}[c]{2in}{
\includegraphics[width=2in]{#4}
\vspace*{-3mm}\caption[]{#5}\label{#6} \
}\end{minipage}
\end{center}
\vspace*{-0.4in}\
\end{figure*}
}
}




\def\dsfigure[#1,#2,#3,#4,#5,#6]{
{
\begin{figure*}
\vspace*{0.2in}\
\begin{center}
\begin{minipage}[c]{3in}{
\includegraphics[width=3in]{#1}
\vspace*{-3mm}\caption[]{#2} \label{#3} \
}\end{minipage}\hspace*{0.5in}\
\begin{minipage}[c]{3in}{
\hspace*{0.5in}\
\includegraphics[height=3in]{#4}
\vspace*{-3mm}\caption[]{#5}\label{#6} \
}\end{minipage}
\end{center}
\vspace*{-0.4in}\
\end{figure*}
}
}


\def\dsyfigure[#1,#2,#3,#4,#5,#6]{
{
\begin{figure*}
\vspace*{0.2in}\
\begin{center}
\begin{minipage}[c]{2.5in}{
\includegraphics[height=2.5in]{#1}
\vspace*{-3mm}\caption[]{#2} \label{#3} \
}\end{minipage}\hspace*{0.5in}\
\begin{minipage}[c]{2.5in}{
\includegraphics[height=2.5in]{#4}
\vspace*{-3mm}\caption[]{#5}\label{#6} \
}\end{minipage}
\end{center}
\vspace*{-0.4in}\
\end{figure*}
}
}

\def\dyfigure[#1,#2,#3,#4,#5,#6]{
{
\begin{figure*}
\vspace*{0.2in}\
\begin{center}
\begin{minipage}[c]{3in}{
\includegraphics[height=3in]{#1} 
\vspace*{-3mm}\caption[]{#2} \label{#3} \
}\end{minipage}\hspace*{0.5in}\
\begin{minipage}[c]{3in}{
\includegraphics[height=3in]{#4} 
\vspace*{-3mm}\caption[]{#5}\label{#6} \
}\end{minipage}
\end{center}
\vspace*{-0.4in}\
\end{figure*}
}
}

%%%%%% DOUBLE FIGURE Y
\def\dyoldfigure[#1,#2,#3,#4,#5,#6]{
{
\begin{figure*}
\vspace*{0.2in}\
\begin{center}
\begin{minipage}[c]{3in}{
\epsfysize=2.0in\
\hspace{0.5in}\
\epsfbox{#1}
\vspace*{-3mm}\caption[]{#2} \label{#3} \
}\end{minipage}\hspace*{0.25in}\
\begin{minipage}[c]{3in}{
\epsfysize=2.0in\
\hspace{0.5in}\
\epsfbox{#4}
\vspace*{-3mm}\caption[]{#5}\label{#6} \
}\end{minipage}
\end{center}
\vspace*{-0.4in}\
\end{figure*}
}
}

%%%%%% DOUBLE FIGURE Y IN A COLUMN!!
\def\cfiguredouble[#1,#2,#3,#4]{
\begin{figure}
\vspace*{0.2in}\
\begin{center}
\begin{minipage}[c]{1.5in}{
\epsfxsize=1.5in\
\epsfbox{#1}
}\end{minipage}\hspace*{0.1in}\
\begin{minipage}[c]{1.5in}{
\epsfxsize=1.5in\
\vspace{0.1in}\epsfbox{#2}
}\end{minipage}\vspace*{-0.10in} \caption[]{#3}\label{#4}
\end{center}
\vspace*{-0.4in}\
\end{figure}
}


%%%%% Single programmable size figure
\def\wpfigure[#1,#2,#3,#4]{
\begin{figure*}
\vspace*{4mm}
\begin{center}

\includegraphics[width=#4]{#1} 

\vspace*{-3mm}\caption[]{#2
} \label{#3}

\vspace*{-5mm}
\end{center}
%\horizbar
\end{figure*}}

%%%%% Single programmable size figure, rotated
\def\wprfigure[#1,#2,#3,#4,#5]{
\begin{figure*}
\vspace*{4mm}
\begin{center}

\includegraphics[width=#4, angle=#5]{#1} 

\vspace*{-3mm}\caption[]{#2
} \label{#3}

\vspace*{-5mm}
\end{center}
%\horizbar
\end{figure*}}




%%%%% Adjacent, programmable-width figures, slid vertically by 9th
%%%%% parameter
\def\DoubleFigureWSlide[#1,#2,#3,#4,#5,#6,#7,#8,#9]{
\begin{figure*}
\vspace*{#9}
\begin{center}
\begin{minipage}{#4}
\includegraphics[width=#4]{#1}
\vspace*{-3mm}\caption{#2
}\label{#3}
\end{minipage}
\hspace{2em}
\begin{minipage}{#8}
\includegraphics[width=#8]{#5}
\vspace*{-3mm}\caption{#6
}\label{#7}
\end{minipage}
\vspace*{-5mm}
\end{center}
\end{figure*}
}


%%%%% Adjacent, programmable-width figures
\def\DoubleFigureW[#1,#2,#3,#4,#5,#6,#7,#8]{
\begin{figure*}
\vspace*{0in}
\begin{center}
\begin{minipage}{#4}
\includegraphics[width=#4]{#1}
\vspace*{-3mm}\caption{#2
}\label{#3}
\end{minipage}
\hspace{2em}
\begin{minipage}{#8}
\includegraphics[width=#8]{#5}
\vspace*{-3mm}\caption{#6
}\label{#7}
\end{minipage}
\vspace*{-5mm}
\end{center}
\end{figure*}
}



\def\DoubleFigureWHack[#1,#2,#3,#4,#5,#6,#7,#8]{
\begin{figure*}
\vspace*{0in}
\begin{center}
\begin{minipage}{3in}
\includegraphics[width=#4]{#1}
\vspace*{-3mm}\caption{#2
}\label{#3}
\end{minipage}
\hspace{2em}
\begin{minipage}{3in}
\includegraphics[width=#8]{#5}
\vspace*{-3mm}\caption{#6
}\label{#7}
\end{minipage}
\vspace*{-5mm}
\end{center}
\end{figure*}
}






%%%%%% DOUBLE FIGURE
\def\ddcfigure[#1,#2,#3,#4]{
\begin{figure*}
\vspace*{0.2in}\
\begin{center}
\begin{minipage}[c]{3in}{
\includegraphics[height=3in]{#1} 
}\end{minipage}\hspace*{0.5in}\
\begin{minipage}[c]{3in}{
\includegraphics[height=3in]{#2} 
}\end{minipage}\vspace*{-0.10in} \caption[]{#3}\label{#4}
\end{center}
\vspace*{-0.4in}\
\end{figure*}
}

\def\ddcfigureSlide[#1,#2,#3,#4,#5]{
\begin{figure*}
\vspace*{#5}\
\begin{center}
\begin{minipage}[c]{3in}{
\includegraphics[height=3in]{#1} 
}\end{minipage}\hspace*{0.5in}\
\begin{minipage}[c]{3in}{
\includegraphics[height=3in]{#2} 
}\end{minipage}\vspace*{-0.10in} \caption[]{#3}\label{#4}
\end{center}
\vspace*{-0.4in}\
\end{figure*}
}

\def\cxfigure[#1,#2,#3]{
\begin{figure}
\vspace*{4mm}
\begin{center}
 
\epsfxsize=2.5in\
\epsfbox{#1}\
 
\vspace*{-0.10in}\caption[]{#2
} \label{#3}
 
\vspace*{-5mm}
\end{center}
%\horizbar
\vspace*{-2mm}
\end{figure}}

\newenvironment{panefigure}{\begin{figure}\begin{center}}{\end{center}\end{figure}}

\newcommand{\pdfpane}[3]{
\begin{minipage}{#1}
\begin{center}
\includegraphics[width=#1]{#2}\\(#3)
\end{center}
\end{minipage}
}


\newcommand\x[1]{$times$}
\newcommand\fixme[1]{{\color{red}#1}}

\newcommand\Ravan[1]{Raven}
\newcommand\blackBox[1]{Raven}



\begin{document}

\title{Raven: A Data-Driven Approach for Automatically Generating Heterogenous Processors}
%\author{}
%\author{Mark Gahagan\\UC San Diego\\mgahagan@cs.ucsd.edu\and Jack Sampson\\UC San Diego\\jsampson@cs.ucsd.edu\and Vinicius Petrucci\\UC San Diego\\vpetrucci@cs.ucsd.edu\and Lingjia Tang\\UC San Diego\\lingjia@cs.ucsd.edu\and Jason Mars\\UC San Diego\\mars@cs.ucsd.edu}
\date{}
\maketitle

\thispagestyle{empty}

\begin{abstract}
Heterogeneous CMPs with two types of cores are now available as
commodity products, and recent proposals argue for increasingly
heterogeneous CMPs as a means to continue to improve
energy efficiency. However, both designing and exploiting increasingly
heterogeneous processors raises many challenges. Design choices for
resource heterogeneity span a vast design space and must balance
diversity commensurate with workload variability without becoming
intractable to verify and construct. At run time, scheduling
applications on the appropriate core is key to profiting from
heterogeneity.

In this paper, we present \Ravan{}, an analytical model for tailoring
heterogeneous processor resources to a target workload based on
architecture-independent properties of the applications. This is done by
pre-selecting the architecture features that can be adjusted as well as
selecting the range of those adjustments. We show that
8 primary features are sufficient to guide both core feature
selection and runtime placement. Using \Ravan{}, we produce heterogeneous
platforms that are an average of 10\% more energy efficient than comparable
two-way heterogeneous platforms and homogeneous processors alike.
\end{abstract}

\section{Introduction}
\label{sec:introduction}
%\subsection{Multi-Core Problems}
%Discuss multi-core architectures and the problems with power that have come up% as a result of evolving tech

The demise of Dennardian scaling~\cite{Dennard74-JSSC-MOSFET_Scaling}
has lead processor designers to increasingly shift their focus from
raw, per-core performance toward energy efficiency as a primary
metric. This transition has given rise to a multitude of multicore
designs across many domains. Compared to their uniprocessor
predecessors, the more modestly aggressive cores in modern multicore
processors reap efficiency benefits not only from avoiding less
rewarding regions of the superlinear relationship between peak core
power and peak core performance, but also from the fact that there are
many applications (e.g. SPECINT-like desktop programs) which will only
rarely approach peak core utilization.

Recent proposals~\cite{Lukefahr12-MICRO-CompositeCores,variable2011multi} and products~\cite{ARM11-WhitePaper-BigLittle} aim to
further capitalize on the latter of these effects by increasing
processor heterogeneity. On a heterogeneous platform, each application
can run on the least energy-expensive processor that still
allows it to achieve a high fraction of its potential performance,
increasing overall efficiency. However, the space of all potential
heterogeneous processors is quite large, even when restricting the
options to traditional core types. It is not yet clear how best to
select the constituent cores in a heterogeneous CMP, nor how best to
map incoming applications to those cores.

There are two prominent schools of thought for heterogeneous design
strategies, but much room remains in the middle for new
approaches. Current designs, such as ARM's
Big.LITTLE~\cite{ARM11-WhitePaper-BigLittle}, draw inspiration from
early work~\cite{Kumar03-SIHM,Kumar06-PACT-SIHM,Kumar04-SIHM} on
heterogeneity that advocates limiting design costs by combining
previously developed architectures. This, however, may limit the energy
savings of the system, because it maps applications to cores whose
resource diversity owes more to temporal changes in design
restrictions than to a concerted effort to more efficiently serve
particular classes of current or future applications. At the other
extreme, proposals such
as~\cite{Clark05-ISCA-CustomISA,Clark08-ISCA-VEAL,Goulding11-IEEEMICRO-GreenDroid,Venkatesh10-ASPLOS-CCores}
call for domain or even application-level specialization of cores,
which maximizes energy efficiency, but vastly increases design effort
and may produce systems that are overly sensitive to changing
workloads. Taking insight from both ends of the design spectrum, the
intuition is that there will be profitable ground in the middle:
heterogeneous systems that consist of general purpose cores with
traditional components, but where each core in the collection can
optimize its performance/energy tradeoffs for a subset of
applications.

In this paper, we present \blackBox{}, a high-level design approach
for automatically selecting the degree and dimensions of heterogeneity
in a CMP based on the architecture-independent features of a target
workload. \blackBox{} constructs processor configurations out of a
modest sized library of fixed components, but with many possible
combinations. Lifting the restriction that every core be well-suited
for executing the entire workload allows \blackBox{} to select
individual cores with non-standard configurations, e.g. cores lacking
a floating point unit and large caches due to the requirements of part
of the workload, while still effecting general application support through the union of
cores selected in a particular design. With \blackBox{} we can quickly
predict what set of cores would be needed in a single CMP to exploit
heterogeneity within a workload and to produce multiple CMP designs to
better exploit heterogeneity among workloads from different domains.

At a high level, \blackBox{} works as follows: \blackBox{} considers a
design space spanning 8 dimensions including cache
size, issue width, and ALU allocation. We train on a small number
of applications across a subset of the configurations within that
design space. For each training application or application phase, we
collect a vector of architecturally independent features
gathered from MICA~\cite{Hoste07-IEEEMICRO-MICA} and the associated
energy-performance-area three-tuple on a given core. To design a
particular heterogeneous multiprocessor, we pass the architecturally
independent feature vectors of a target workload into a trained
\blackBox{}, as well as the maximum allowed degree of heterogeneity,
\textit{K}. \blackBox{} then partitions the input applications into up
to K groups and, for each group, estimates the energy and performance
of the group across all points in the design space. Finally,
\blackBox{} picks the optimal performance/watt configuration across
each group, selects that processor for inclusion in the CMP, and
provides a distance-based mapping function between the
architecture-independent features of any current or future application
and the predicted-preferred core type.

The contributions of this work include the following:
\begin{itemize}
\item We describe \blackBox{}, a design approach for automatically
  selecting an energy-efficient set of heterogeneous cores, given an
  architecturally independent description of a workload.
\item We show that \blackBox{} designs with greater degrees of
  heterogeneity offer superior energy efficiency compared to
  homogeneous or limited heterogeneity designs, improving on existing
  designs by up to 18\%.
\item We show that there is sufficient variation among workloads, and that
  different workloads warrant different heterogeneous solutions.
\item We explore the sensitivity of \blackBox{} designs to workload
  variation and show that \Ravan{} performs well against other potential
  solutions.
\end{itemize}

The rest of the paper proceeds as
follows. Section~\ref{sec:background} provides background on the
allure of and challenges posed by heterogeneous architectures, and
discusses related work. Section~\ref{sec:design} presents our design
for \blackBox{} and Section~\ref{sec:validation} validates our
approach. Section~\ref{sec:evaluation} evaluates the heterogeneous
processors that \blackBox{} produces and Section~\ref{sec:conclusion}
concludes.

% LocalWords:  Dennardian multicore uniprocessor superlinear SPECINT
% LocalWords:  CMP ARM's tradeoffs ALU tuple



\section{Background and Related Work}
\label{sec:background}
\label{sec:related}



The fundamental appeal of heterogeneous architectures lies in the fact
that workload demands are inherently heterogeneous at several levels:
Applications in different domains have different characteristics,
different applications within a domain differ in their resource
bottlenecks~\cite{Vasilieospeakpowermicro2010}, and even within a
single application, different phases provide different
power/performance tradeoffs.  

\subsection{Heterogeneity in General Purpose Multiprocessors} 
The body of work on single-ISA heterogeneous multicore
processors~\cite{Kumar03-SIHM,Kumar06-PACT-SIHM,Kumar04-SIHM} investigated the power/performance tradeoffs
for asymmetric CMPs. By transitioning on phase boundaries between
aggressive and simple cores, these designs were able to save up to 50\%
in power budget for a performance reduction of only 10\%. That all said, the
selection of which particular cores to include in a particular CMP
design for a particular workload was not considered in depth. 

More recently there have been efforts to place heterogeneous
designs on physical silicon.~\cite{ARM11-WhitePaper-BigLittle} These
processors combine two existing architectures with different execution
models, but are designed to work as a heterogeneous unit, varying the
execution on the processor based on the needs of the application. Some
take this concept one step further by combining as many architectural
components as possible to try and develop a single heterogeneous 
core~\cite{Lukefahr12-MICRO-CompositeCores}; capable of switching 
between a large out-of-order engine and a small in-order engine
depending on the IPC of the application being run. \Ravan{}
improves on previous approaches relying on collections of existing
processor models by automatically selecting the appropriate degrees of
asymmetry to best exploit a workload. The range of asymmetry is only
limited by the number of architecture features one desires to change for
a particular core design.

\subsection{Heterogeneity in specialized architectures}
As power constraints tighten, designers are increasingly integrating
specialized coprocessors into general-purpose architectures.  GPUs are
an especially common addition, and are now found on-chip in everything
from cell phones to servers. Many recent
efforts~\cite{Kim09-MICRO-Qilin} attempt to harness these
heterogeneous platforms with language extensions like
CUDA~\cite{scalableCUDA} and streaming frameworks such as
Brook~\cite{brook}, but they focus primarily on highly-parallel code
and loosely-coupled execution models. Even flexible heterogeneous
processing frameworks such as Intel's EXOCHI~\cite{EXOCHI} face
deployment challenges in scaling to greater degrees of heterogeneity:
EXOCHI's uniform abstraction for sequencing execution across
heterogeneous execution engines still requires specialized compilers
for each piece of target hardware. In contrast, \Ravan{} is designed
to run existing, unmodified binaries and maintain a focus on a single-ISA
design, rather than relying on multiple programming models to achieve
maximum efficiency.

Recent efforts on generating internally diverse processors have
focused on automating the production and use of specialized
coprocessors~\cite{Venkatesh10-ASPLOS-CCores,sampson-HPCA-ECOCORES}.  These
automatically-generated coprocessors do not achieve the performance of
hand-crafted accelerators, but they are very energy-efficient and can
target nearly-arbitrary code. \Ravan{} on the other hand focuses on
remaining a general purpose platform, allowing for any application that is 
designed for the underlying ISA to be executed. At the same time, it does not 
require processors to be so specific that they have little use outside of 
the applications they were trained under, and indeed this is a design choice
that tends to be avoided when possible.

\subsection{Heterogeneous scheduling}
Scheduling a heterogeneous processor presents unique challenges but is
essential in order to provide energy benefits from its varied set of cores. 
There are varying approaches on how best to perform this scheduling, from monitoring metrics
during execution~\cite{PIE} to developing a programming system that dynamically
schedules applications based on running the application through a virtual machine 
first~\cite{Kim09-MICRO-Qilin}. These methods rely on focusing on the monitoring of
applications during runtime, and making changes in scheduling based on
the application's performance on the system.

One other way that scheduling is performed is by gathering the architecture signatures of an 
application to determine the best fit out of a given set of cores.~\cite{Shelepov09-SIGOPS-HeteroScheduling}
Rather than focus on a phase-driven analysis, scheduling is performed
by looking at the typical behavior of the application given a set of parameters.
\Ravan{} deploys a similar technique, using architecture independent metrics
in order to make informed decisions on the type of core that would be a good
fit for a given application. In either case it will need an efficient
scheduling algorithm to take advantage of the cores that are designs so research
in this area is important to the overall health of the concept of a heterogeneous
processor.



\subsection{Comprehensive Prediction}
For the most part, the desire for a comprehensive prediction and core 
generation scheme for heterogeneous processors is a relatively new one.
Efforts to date have been focused on making well-established changes to 
processor configurations or combining accelerators with current
general-purpose cores to achieve the goal of heterogeneity as discussed
earlier. The goals have changed as well; previously they been focused on raw performance 
improvements~\cite{Kumar06-PACT-SIHM} or attempting to cluster applications themselves to aid
in the design of broader accelerator-based systems~\cite{10x10}, but not necessarily
in a single-ISA framework. \Ravan{} attempts to solve this issue, providing a 
prediction model that can be broadened to any number of architectural features as needed while
using architecture independent metrics to aid in the construction of heterogeneous cores.



% LocalWords:  tradeoffs ISA multicore CMPs CMP IPC coprocessors GPUs
% LocalWords:  CUDA EXOCHI EXOCHI's runtime

\section{\blackBox{} Design}
\label{sec:design}

\wfigure[lib/figures/Workflow.png, {\figtitle{Procedural Flow of Raven Design:} Raven is designed so that processor designers can tailor their training sets to the types of applications they expect to see on their processors, then continue to train during subsequent core design runs.}, fig:prism-designflow]

\blackBox{} is a regression-based model that we have designed to
perform two tasks. First, given a workload, \blackBox{} must be able
to accurately select a covering set of core designs that collectively
constitute a general purpose, energy efficient CMP. Second, given a
previously generated CMP and a new application, \blackBox{} must
accurately predict on which core on the CMP the application will most
efficiently run.

Given any non-trivial set of heterogeneity dimensions, the search
space for determining the best set of heterogeneous cores for a given
workload gets very large very quickly. Moreover, even if an exhaustive
search were tractable, real processors must execute applications
developed after the processor's introduction and run existing
applications on new inputs. Thus, the application phases used to train
\blackBox{} are described solely in terms of their
architecture-independent features, and in Section~\ref{sec:evaluation}
we will closely examine the sensitivity of \blackBox{}-derived
processors to workloads divergent from the original training set.

Below, we discuss the key steps in building a fast and accurate model
for accurately selecting and mapping among heterogeneous cores.

\boldparagraph{Defining a design space} 
One of the first steps in designing \blackBox{} revolves around
determining the architecture features, or knobs, that the user will
have available in the processor generation search space.  Since the
goals for a heterogeneous processor tend to revolve around some
combination or power, energy, and/or area, care should be taken to
select a set of knobs that addresses at least some of these concerns.

High impact changes, such as issue width and size and type of the
pipeline, are currently the focus of current heterogeneous 
design efforts, since they can greatly influence
both the energy consumption and the performance of a given core. Thus, 
they should nearly always be included in any search space. Other 
aspects such as cache structure and functional units should be strongly considered 
as well. Smaller power draws such as branch prediction
and prefetchers can be modeled as well should the need arise, but
 the total number of dimensions and number of choices must
remain modest in order to keep both hardware library design effort and
model training times tractable.

Once the knobs are selected, one of the first steps to perform is to
acquire power and area information based on a target
architecture. These are numbers that are used by \blackBox{} in
performing comparisons in the various knobs options and thus should be
as accurate as possible to the architecture one is building on. For
this, we have used the McPAT~\cite{mcpat} power and area modeling
framework to calculate the $\Delta$-power/area changes in a processor
depending on the knob settings (e.g. toggling the number of Integer
ALUs while keeping all other values constant). This will aid in
getting power and area information that is reasonably accurate for
what the modeling data would provide without the need to run the
framework for every single possible configuration.

\boldparagraph{Selecting an optimization function} There are many
optimization functions one may wish to choose for an approach like
\blackBox{}. For expedience in evaluation, our current
implementation of \blackBox{} optimizes for performance/Watt, but it
could easily be extended to take the optimization function as an input
parameter.

\boldparagraph{Selecting input parameters} In order to make
\blackBox{} scalable and easily usable, there needs to be a concise
representation of the types of workloads the user will want to try to
optimize for. The reason for this is that it influences the rest
of the \blackBox{} process as well as gives an initial design space
for our training sets and equation formulation. We choose a set of
architecture-independent features because these can be easily gathered
for any application to be tested on \blackBox{}. We acquire these
architecture-independent datapoints using the Intel Pin~\cite{pin} program
and specifically the MICA pintool~\cite{Hoste07-IEEEMICRO-MICA} for all applications
within the workload. The goal would then be to pick the relevant data
to the knobs that have been selected and select a training set based
on these values. These values are also saved as they are inserted into
the training set to help generate the performance equations.

\boldparagraph{Generation of Training Data and Regression} The
accuracy of \blackBox{}'s predictions will depend on the
representativeness of its training set. 
Modeling data as accurately as possible is key to getting accurate
heterogeneous cores. To this end, we generate
simpoints~\cite{Sherwood01-PACT-Simpoint} for each of the applications
in the workload irrespective of whether the apps are used in training
either the regression or the processors. It is well understood that
applications generally have multiple phases, so in order to have the
most accurate representation of the workload simpoins will need to be
gathered for each application in the workload. For all
simulations we use gem5~\cite{gem5}, and we were able to generate the
basic block vectors necessary for simpoint analysis as well as create
the checkpoints necessary to speed up future runs. For all application
runs we select a simpoint and run for 100 million instructions, and
this is performed for all runs, be they for training set generation or
processor validation.

In order to generate the training set data, a set of applications
needs to be selected that best represents the workload as a
whole. This can be determined by examining the data from MICA and 
looking for varying application data so the training set can cover an adequate
surface space. Gem5 simulations are then run on the training applications, 
varying one of the knobs at a time while keeping the rest of the processor constant. 
Care should be taken to select some sort of default or "Midway" processor
that can serve as a baseline for future calculations, and as a
processor configuration that this training data revolves around. 

Once each application has been processed for both the simulation-based
processor data and the architecture independent variables, we combine
the two into a single training set and send it through a linear
regression program within the Weka machine learning
suite~\cite{weka}. The standard analysis will result in a single
equation that contains both the various knobs (x$_{1}$,
x$_{2}$,....x$_{j}$) and the various arch-independent data points
(y$_{1}$, y$_{2}$,....y$_{k}$) resulting in:
\begin{equation}
perf = \beta _1x_1+\beta _2x_2+...+\beta _1y_1+\beta _2y_2+...+\alpha 
\end{equation}
This performance equation will estimate the performance across the knob space 
effectively while holding the application data points fixed for any particular
core fitting exercise. While this may work for basic situations, additional 
complexities do arise in constructing these regression and we will address this
in section \ref{sec:validation}.
 
\boldparagraph{Processor Prediction} 
Once the above steps are completed, \blackBox{} is ready to generate
processors.  A workload can be selected at either a one process per
core granularity or on the cluster level. Clustering is generally 
desirable in the case where there are more applications than available
cores. To perform this, we run a K-means clustering on
the application characteristics, but any clustering algorithm will
do as long as the number of clusters is no more than the number of cores. 
In this particular case, we will then use these "meta apps" as the
input for \blackBox{} rather than relying on the application that is
the closest to the cluster median. The list of inputs into \blackBox{}
are as follows:


\begin{itemize}
\item McPAT data for the $\Delta$-changes in the area and power of the processor
\item Linear regression for performance
\item EITHER the application specific or cluster metadata for each core to be tested
\item An equation denoting the goal (i.e. maximizing for performance/watt, performance/area) etc.
\item The list of knobs that are being explored and the values within the knobs
\item The configuration of the "default" core
\item Number of processors to validate after \blackBox{} analysis completes
\end{itemize}

After setting up the necessary data structures, \blackBox{} performs
an entire sweep of the knob search space, calculating the projected
performance, area, and power for each possibility. The area and power
metrics are determined by increasing the projected value based on the
introduction of more components, while the performance is determined
using the regression equations given in the input. We perform this
exhaustive search because estimating for these knobs is still much
quicker than performing individual simulations, while also avoiding
any pitfalls of pruning the search space unnecessarily. Some basic pruning
decisions are made (e.g. there is little need to model varying instruction 
window sizes on an in-order processor), but nothing that is a reasonable and
unique core is left out. Once this search is
complete, \blackBox{} will select the top processors that achieve at least
the same predicted performance/watt as the default core, 
then outputs the results to a file for further analysis.

The cores that were selected by \blackBox{} are run through gem5 to 
get accurate performance measurements beyond that of what the model 
can perform. These numbers are coupled with McPAT data generated from the
same processors to allow us to select the best core within the limited
search space for the given goal. \blackBox{} then reports
the processors for each application/cluster to the user. One final
step consists of adding the results of the gem5 runs to the training
set, which allows the user to continuously tune the regression to
their workloads without having to retrain from scratch every
time. While in the end we do perform gem5 simulations, they remain in
the order of tens of simulations versus the order of tens of thousands
for even basic heterogeneous core search spaces. 

\section{Refinement and Setup of \blackBox{}}
\label{sec:validation}
In this section we describe the refinements made to \blackBox{} in order
to tailor the tool for our purposes, and provide some insight on whether
adjustments would need to be taken in similar situations. We then look at 
the steps we took to set up our experiments, thus validating that our design 
is sound and allowing for the evaluation of our \Ravan{} cores against other hardware. 

\subsection{Refinements}
\label{sec:refinements}
One of the key heuristic changes made was acknowledging that a single regression
equation for performance can lead to incorrect predictions based on training 
set data. One key example that we encountered was the introduction of floating 
point applications and a floating point (FP) unit knob to the knob list. While 
the regression would accurately penalize FP operations that would run without
an accompanying FP ALU in the processor, it would also assign the same penalty 
for integer only workloads that would try to save power by eliminating the
unit. This sort of interdependence was not unique to integer vs floating point,
but it had one of the largest impacts on the predicted performance. The solution 
to this involved two parts. First, The regression components were separated: 
requiring that any directly FP-dependent variables (both from knobs and 
independent variables) were omitted from the int-regression and the reverse was
true for the FP-regression. Then, because the int regression would not over-provision
performance due to a lack of knowledge, the two regressions are weighted based on
the ratio of floating point operations to total operations.
\begin{equation}
perf = ((1-FP_{inst})*Reg_{int} - FP_{const})+ (FP_{inst})*Reg_{fp}
\end{equation}

This new equation gave much more accurate numbers for our out-of-order execution
predictions, but the equation was still lacking for in-order executions. The model 
was not accurately accounting for memory-bound applications that were
spending much of their time waiting for data from DRAM. Once again, the regression
gave a performance boost for out-of-order execution, but did not take this 
particular scenario into account, due in part to the large variations in cache
behavior between the applications. One architecture independent metric used
was the average reuse distance of memory addresses, and noted that there was
a direct correlation between this metric and the performance penalty of 
switching to in-order cores. The solution was then add a component to the regression,
where we add a fraction of the data reuse variable only to in-order processors. 
This adjustment solved the issues with in-order performance while at the same
time not affecting out-of-order performance, which was not affected by this issue.

\begin{equation}
Reg_{x} = Reg_{x} + (DataReuse * \beta) * ((inOrder == 1?) 0 : 1)
\end{equation}


Lastly, it was noted that the regressions did not use all of the architecture
independent variables or had $\beta$ coefficients that were negligible. We decided
to keep these variables rather than discard them in order to create more informed
clusters during the clustering phase of \blackBox{}.

\begin{table}[tl]
   \centering
   \begin{tabular}{|p{1.5in}|p{1.5in}|}
	\hline
	\textbf{Component} & \textbf{Settings}\\
	\hline
	L1 Cache & 16KB/2-way I-D, 32KB/2-way I-D, 64KB/4-way I-D\\
	\hline
	L2 Cache & 64KB, 256KB, 1MB\\
	\hline
        Int ALUS & 1, 3, 6\\
	\hline
	Mul/Div ALUS & 1, 2\\
	\hline
	FP Units & 0, 1, 2\\
	\hline
	Instruction Window Size (for OoO models) & 64, 128, 200\\
	\hline
	Issue Width \& Execution Model & 1-IO, 2-IO, 4-IO, 4-OoO, 8,OoO\\
	\hline
   \end{tabular}
   \caption{Knob Selection for Raven Experiments}
   \label{table:knobs}
\end{table}

\begin{table}[tl]
   \centering
   \begin{tabular}{|p{0.7in}|p{0.8in}|p{1.4in}|}
	\hline
	\textbf{Application} & \textbf{Suite} & \textbf{Purpose}\\
	\hline
	gcc & SPECInt 2006 & Code Compilation\\
	\hline
	libquantum & SPECInt 2006 & Quantum Simulation\\
	\hline
        namd & SPECFP 2006 & Molecular Dyanmics\\
	\hline
	milc & SPECFP 2006 & Lattice Computation\\
	\hline
	freqmine & PARSEC 2.1 & Itemset Mining\\
	\hline
	swaptions & PARSEC 2.1 & Monte Carlo Sim\\
	\hline
	fft & MiBench & Fast-Fourier Transform\\
	\hline
	stringsearch & MiBench & String Comparison\\
	\hline
   \end{tabular}
   \caption{Training Set Applications for Raven Experiments}
   \label{table:trainingset}
\end{table}

\begin{center}
\begin{table*}[ht]
{\small
\hfill{}
\begin{tabular}{|l|c|c|c|c|c|}
\hline
\textbf{Component}&\textbf{Midway}&\textbf{8-OoO}&\textbf{1-IO}&\textbf{Tag-Team:Large}&\textbf{Tag-Team:Small}\\
\hline
\textbf{Width and Execution Model}&4-OoO&8-OoO&1-IO&4-OoO&2-IO\\
\hline
\textbf{L1 Cache}&32 KB-2 Way&64 KB-4 Way&16 KB-2 Way&32 KB-2 Way&32 KB-2 Way\\
\hline
\textbf{L2 Cache}&256 KB&1 MB&64 KB&1 MB&1 MB\\
\hline 
\textbf{Int ALUS}&3&6&1&3&1\\
\hline
\textbf{Mul/Div ALUS}&1&2&1&1&1\\
\hline
\textbf{FP Units}&1&2&1&1&1\\
\hline
\textbf{Instruction Window}&128&200&N/A&128&N/A\\
\hline
\end{tabular}}
\hfill{}
\caption{Baseline Processors (The \emph{Midway} processor is what graphs and data are normalized to for all metrics.)}
\label{table:procsundertest}
\end{table*}
\end{center}

\subsection{Setup of \blackBox{}}

With the above adjustments in place and the component design complete, we will
now describe our setup for the experiments and evaluation to follow. The 
attempt was to have as broad of a spectrum of applications as possible. The idea 
is to perform both breadth testing by looking at a processor that was
developed across all possible suites as well as doing some analysis on processors
specifically designed with one processor in mind. As a result, we drew from 
the desktop, embedded, and HPC communities, selecting from the following 
benchmark suites:

\begin{itemize}
\item Desktop: SPEC2006~\cite{spec2006}
\item HPC: Parsec 2.1~\cite{parsec}
\item Embedded: MiBench~\cite{mibench} and Coremark~\cite{coremark}
\end{itemize}

The next step was determining the knobs that we would iterate against. Since
current heterogeneous designs focus on issue width and execution model, it seems
like that would be a reasonable starting point. Our decision points for which
issue width and execution model combinations to analyze were based on previous
studies that look at the energy-performance tradeoffs of these types of 
processors~\cite{horowitz}. The next focus is on high power components that 
could be scaled in various ways and have meaningful impacts on performance
We ended up selecting the various ALUs, cache sizes, and the instruction
window of the out-of-order models as our additional knobs to test with. 
Table \ref{table:knobs} shows our detailed selection and settings for each knob.

The selection of the knobs also allowed us to see what kinds of architecture
independent variables to use. It was clear that we needed a breakdown of 
what instructions were actually executing as well as detailed information
related to the memory subsystem. We wanted to select a series of metrics
that gave a detailed outlook as to how the application performs, even if
some of the metrics are used only for clustering purposes. In the end we 
selected the following independent variables: 

\begin{itemize}
\item Integer, Floating Point, Branch, and Memory Access Instruction Rates
\item How much memory passes through the processor in a given simpoint (its footprint)
\item Average data reuse and the percentage of data reuses in a short distance
\item Register operand average
\item Average register producer-consumer rate 
\end{itemize}

With this information we can collect both our McPAT data and select
a base processor to test the rest of our candidate processors against. While a detailed 
description of the processor can be found in table \ref{table:procsundertest}
we essentially selected the midpoints for all of our knobs and focused on a
4-wide OoO processor in order to provide an aggressive performance/watt target that
did not penalize against compute-bound workloads (much like current consumer processors)
This processor, henceforth known as Midway, is used to normalize our 
performance-energy-area tuple so we can focus on improving from a hypothetical
homogeneous core model making basic decisions on optimizing for performance/watt
rather than try to normalize around an extreme processor that may or may not be trying
to achieve that goal.

Finally, as far as the training set goes, we selected two representative applications from
Parsec and MiBench, as well as 2 from SPECInt and SPECFP. Table \ref{table:trainingset}
highlights the processes chosen as well as their basic functionality. The goal 
was to give the training set a wide variety of potential scenarios so it can 
make reasonable approximations for new applications that may not fall exactly 
within the training set. With the heuristic adjustments made to the performance
equation discussion in Section \ref{sec:refinements} the training set proved to
be adequate for our evaluation.







% LocalWords:  CMP prefetchers ALUs scalable datapoints pintool Weka
% LocalWords:  representativeness simpoints simpoint



\section{Evaluation}
\label{sec:evaluation}

In this section we will discuss the results of experiments performed
to determine the viability of the \Ravan{} model. We first examine
using a general-purpose target workload with \blackBox{} to construct
the \emph{Raven-G} CMP and evaluate Raven-G across the applications of
several benchmarking suites.  We then explore how optimizing for a
domain-specific workload affects core selection and how sensitive
those processors (\emph{Raven-M} and \emph{Raven-S}) are to new
applications entering their workload. Lastly we look at workload-level
statistics to show the overall effectiveness of \Ravan{} across all
test cases.

\subsection{Experiment Set-Up}

\begin{table}[h!]
   \centering
   \begin{tabular}{|p{1.5in}|p{1.5in}|}
	\hline
	\textbf{Processor} & \textbf{Target Applications}\\
	\hline
	\textbf{Raven-General (Raven-G)} & basicmath, blackscholes, canneal, coremark, mcf, \emph{milc}, xalan, zeusmp\\
	\hline
	\textbf{Raven-MiBench (Raven-M)} & adpcm, basicmath, blowfish, coremark, crc, \emph{fft}, \emph{stringsearch}\\
	\hline
        \textbf{Raven-SPEC (Raven-S)} & \emph{gcc}, \emph{libquantum}, mcf, \emph{milc}, namd, povray, soplex, xalan\\
	\hline
   \end{tabular}
   \caption{Workloads for Evaluation Experiments (Applications in training set are \emph{italicized})}
   \label{table:targets}
\end{table}


In order to test both the breadth and the depth of Raven's
capabilities, we have constructed three target workloads, each
containing eight applications. Table~\ref{table:targets} lists the
applications in each workload. While some of the training set
applications are present in the target sets, the goal was to create
processors based on a majority of new programs not present in the
regressions. We cap maximum heterogeneity in our Raven cores to four
core-types per \Ravan{} CMP. Since we have twice as many applications
in the workload, we perform K-Means clustering with K=4 to produce up
to four sets of meta-application characteristics, and use those values
as the input to \blackBox{}.  We then evaluate the selected \Ravan{}
CMPs and our baseline CMPs on both the target workloads, and workloads disjoint from the target set constructed from other benchmarks in the same domain.

Table~\ref{table:procsundertest} details the baseline designs we
compare our \Ravan{} cores against. We use these cores as the baseline
in all experiments.  They include the Midway processor that we use to
normalize our metrics, an 8-wide out of order (OoO) core featuring the
maximal resource allocations in all dimensions in our design space, a
1-wide in-order (IO) core featuring the minimum resource configuration
in our design space, and a two processors combination,
\emph{tag-team}, that reflects current design philosophies for
heterogeneous design by coupling a 4-wide OoO core and a 2-wide IO
core. For tag-team results, we assume that every application is always
scheduled to the core on which it achieves superior performance/watt.

We evaluate the different designs across several workloads using
performance/watt as our optimization metric. The performance/watt
metric is a good measure of fitness because it simultaneously offers
an easy intuition on how these processors can either scale up in
throughput or provide savings under a fixed budget.

\ignore{
 as our core energy efficiency metric here 
because one of the goals of heterogeneous processors is to save energy over 
the more "power-hungry" homogeneous cores that dominate the market today. By
showing an improvement in performance/watt over other options we hope to convey
the need to delve deeper into the potential energy-saving opportunities within
an architecture above and beyond the current options being designed today.
}

\subsection{Raven-G: General Workload}

\begin{center}
\begin{table*}[htbp!]
{\small
\hfill{}
\begin{tabular}{|l|c|c|c|c|}
\hline
\textbf{Component}&\textbf{Raven-G:1}&\textbf{Raven-G:2}&\textbf{Raven-G:3}&\textbf{Raven-G:4}\\
\hline
\textbf{Target Applications}&basicmath, blackscholes, zeusmp&milc&canneal, xalan, coremark&mcf\\
\hline
\textbf{Width and Execution Model}&4-OoO&4-IO&4-OoO&1-IO\\
\hline
\textbf{L1 Cache}&32KB-2 Way&32KB-2 Way&32KB-2 Way&16KB-2 Way\\
\hline
\textbf{L2 Cache}&64 KB&64 KB&256 KB&64 KB\\
\hline
\textbf{Int ALUS}&3&3&3&1\\
\hline
\textbf{Mul/Div ALUS}&1&1&1&1\\
\hline
\textbf{FP Units}&1&1&0&0\\
\hline
\textbf{Instruction Window}&64&N/A&64&N/A\\
\hline
\end{tabular}}
\hfill{}
\caption{Processor Cores within Raven-G: Raven for General Workloads}
\label{table:raveng}
\end{table*}
\end{center}

\dcfigure[lib/figures/general-train-perfwatt.pdf,{\figtitle{Raven-G Perf/Watt on Targeted Applications:} We evaluate the performance/Watt of a \Ravan{} processor designed for a general workload. \Ravan{}-G consistently provides better efficiencythan any of our other reference processors on its target workload.},fig:gen-training,lib/figures/general-untrain-perfwatt.pdf,{\figtitle{Raven-G Performance on broad suite:} Applications from all three domains performed favorably on the Raven-G processor as shown here. Efficiency did not suffer greatly running new applications, even when outliers such as lbm are removed.},fig:general-workload]

We start by running \blackBox{} over a general
workload. Table~\ref{table:raveng} shows what the selection model
chooses as the best processors. There are four distinct cores that
differ primarily in two dimensions, reflecting the amount of floating
point computation and the memory access patterns of the applications.
Figure \ref{fig:gen-training} shows the performance/watt results of
running the target workload on all the baseline processors as well as
Raven-G.  Here we can see that across the workload, and for most
applications, \Ravan{}-G provides an efficiency improvement over the
other designs. When suboptimal choices are made, the penalties are
low.

We next investigate whether the Raven-G processor can maintain its
advantage over a larger workload. We select various applications from
across our domain suites to form a ``general'' workload. We map these
applications to the cores in \Ravan{}-G by a simple euclidian distance
metric between the architecture independent feature vector for the
application to be mapped and the mean architecture-independent feature
vector from the cluster of target applications that drove core
selection. The results of this are found in
figure~\ref{fig:general-workload}.

For certain workloads such as lbm, we do exceedingly well due to its
placement on an in-order core, and even in cases where we perform
worse than the Midway core we still outperform all other options,
including the Tag-team heterogeneous core. While \Ravan{}-G is not
always the most efficient design, it achieves its goal of ``general''
purpose by being more efficient on average than the other available
cores. Raven-G, improves performance/Watt efficiency by 9\% over the
Midway core and by 18\% over the tag-team core. While it is less
surprising that we beat the extreme cores, they serve as a reminder
that neither the maximal nor minimal extrema tend to provide superior
efficiency. While the Midway core provides better energy efficiency 
that the Tag-team system, this is largely due to the large caches
present in the heterogeneous system. These caches were designed with
a worst-case scenario with general purpose applications in mind 
where applications need the large cache, something Midway and Raven 
do not account for given the workloads presented.

Collectively these results show that for a general workload, Raven-G
can improve the performance/watt by a measurable amount with very
little overhead in actually generating the core designs. Secondly, it
shows that expanding the asymmetry of heterogeneous CMPs continues to
improve performance/watt beyond current designs. Even simple decisions
like removing a floating point unit for integer based workloads can
have a significant improvement as experiments in McPAT showed a
relatively large leakage wattage (around 0.25 W, which for an in-order
core is a source for significant power savings). 
\subsection{Raven-S and Raven-E: Sensitivity Analysis}
\begin{center}
\begin{table*}[ht]
{\small
\hfill{}
\begin{tabular}{|l|c|c|c|c|}
\hline
\textbf{Component}&\textbf{Raven-M:1}&\textbf{Raven-M:2}&\textbf{Raven-M:3}\\
\hline
\textbf{Target Applications}&adpcm, crc, blowfish, coremark&basicmath, fft&string\\
\hline
\textbf{Width and Execution Model}&2-IO&2-IO&2-IO\\
\hline
\textbf{L1 Cache}&32 KB-2 Way&32 KB-2 Way&64 KB-4 Way\\
\hline
\textbf{L2 Cache}&64 KB&64 KB&64 KB\\
\hline
\textbf{Int ALUS}&1&1&1\\
\hline
\textbf{Mul/Div ALUS}&1&1&1\\
\hline
\textbf{FP Units}&0&1&1\\
\hline
\end{tabular}}
\hfill{}
\caption{Processor Cores within Raven-M: Raven for MiBench Workloads}
\label{table:ravenm}
\end{table*}
\end{center}

\dcfigure[lib/figures/mibench-train-perfwatt.pdf,{\figtitle{Raven-M Perf/Watt on Targeted Applications:} Many of the MiBench benchmarks had similar characteristics, leading to a very lightweight set of cores. While performance hits were greater for some applications over others, there was still a similar trend to the Raven-G case.},fig:mibench-trained,lib/figures/mibench-untrain-perfwatt.pdf,{\figtitle{Raven-M Perf/Watt on New Applications:} Even though the target applications proved to be similar, it is possible for multiple programs that are not the targets for Raven-M to not perform as well as the targeted applications. Since MiBench has low instruction count benchmarks with a limited number of behaviors, subtle changes in the architecture can have large effects on performance.},fig:mibench-untrained]

In order to see the effects of different single-domain suites on
Raven, we decided to run sensitivity analysis experiments using the
MiBench and SPEC benchmarks suites.

First, we tackle MiBench and its embedded system-minded
applications. While these applications were used in the general test
and performed relatively well, there are some key differences when
they are tested alone. Many of these benchmarks have very similar
independent characteristics, so it is more difficult to get a good
sense of the sources of variation in these applications. Due to the
similarities, the clustering to produce \Ravan{}-M ends up with
greater overlap than in \Ravan{}-G, and \blackBox{} select three
processors to cover the four clusters (two clusters mapped to the same
processor), as outlined in table \ref{table:ravenm}.

As a result of the vastly different workload from the general case,
\blackBox{} selects all 2-wide in-order cores. Intuitively, this makes
some sense; a series of embedded benchmarks should be expected to
perform well on lightweight cores and see only limited improvement
from stronger processors. Figure \ref{fig:mibench-trained} shows the
results of running the target applications on the Raven-M
processor. Once again, for most cases \Ravan{}-M provides better
efficiency than the other options, and the patterns are similar to the
the results seen for Raven-G.

\begin{center}
\begin{table*}[t]
{\small
\hfill{}
\begin{tabular}{|l|c|c|c|c|}
\hline
\textbf{Component}&\textbf{Raven-S:1}&\textbf{Raven-S:2}&\textbf{Raven-S:3}&\textbf{Raven-S:4}\\
\hline
\textbf{Target Applications}&povray, soplex, namd&libquantum, milc&gcc, xalan& mcf\\
\hline
\textbf{Width and Execution Model}&4-OoO&4-IO&4-OoO&1-IO\\
\hline
\textbf{L1 Cache}&32KB-2 Way&16KB-2 Way&32KB-2 Way&16KB-2 Way\\
\hline
\textbf{L2 Cache}&64 KB&64 KB&256 KB&64 KB\\
\hline
\textbf{Int ALUS}&3&3&3&1\\
\hline
\textbf{Mul/Div ALUS}&1&1&1&1\\
\hline
\textbf{FP Units}&1&1&0&0\\
\hline
\textbf{Instruction Window}&64&N/A&64&N/A\\
\hline
\end{tabular}}
\hfill{}
\caption{Processor Cores within Raven-S: Raven for SPEC Workloads}
\label{table:ravens}
\end{table*}
\end{center}

\dcfigure[lib/figures/spec-train-perfwatt.pdf,{\figtitle{Raven-S Perf/Watt on Targeted Applications:} We compare \Ravan{}-S to our reference processors over the \Ravan{}-S target workload. Once again, the SPEC benchmarks that are part of the target workload fare better on average on \Ravan{}-S than on the homogeneous cores or on tag-team. The cores selected were very similar to the ones selected for \Ravan{}-G.},fig:spec-trained,lib/figures/spec-untrain-perfwatt.pdf,{\figtitle{Raven-S Perf/Watt on New Applications:} Performance-per-watt variations will occur with new applications, but because SPEC is more general and the target set covered a wider range of applications, the new programs had an impact similar to that seen on the Raven-G processor.},fig:spec-untrained]

Looking at the performance of applications outside the target workload
of the Raven-M processor, the story takes a slightly different turn,
as shown on figure \ref{fig:mibench-untrained}.  Other applications
within MiBench show that \Ravan{}-M continues to do as well or better
than tag-team on average, but not to the degree we saw in the Raven-G
experiment. We also included a separate offender application,
blackscholes, from the PARSEC benchmark suite, to show what might
happen if the workload begins to shift. It is worth noting that
applications that we expect to perform well on in-order cores, such as
lbm, milc, and mcf, were not selected for consideration but if the
workloads shifted in a more memory intensive direction then it is
possible that Raven-M would perform far above the expectations
outlined here. Still, we perform 7\% better than Tag-team, due in
large part to the cache structure in that processor that is not
present in Raven-M.

Lastly, we look at a Raven processor designed solely around SPEC. We
designed our target workload based on previous work on subsetting
SPEC~\cite{subsettingSPEC} in order to try to capture the widest
variety of applications within the benchmark suite with the fewest
applications. In doing so, we ended up with cores that were nearly
identical to the ones found in the Raven-G processor (see table
\ref{table:ravens}). The Raven-S processor shows that it is possible
for applications that have "strong" affinity toward particular setups
(such as desiring a larger caches or floating point unit) to
dominate the decision making process for \blackBox{}.

Figure \ref{fig:spec-trained} shows what happens when we run these
cores on their target workload.  We see a very similar story to the
Raven-G cores with a couple of exceptions.  Thanks to the introduction
of libquantum and the fact it shares a core with milc, the core's
benefit to milc was decreased by \blackBox{}, which led to a relative
performance degradation over the Raven-G case. That being said,
Raven-S still performed better than all other processors for milc, and
libquantum saw significant gains over Raven's competitors.

Once we check for the sensitivity of Raven-S we end up seeing the
opposite end of the spectrum from Raven-M (see figure
\ref{fig:spec-untrained}). Applications like lbm (which do very well
on in-order cores), cause a large spike in the perf/watt capabilities
of many of the processors on the graph. Across the rest of the
applications we see similar performance as before, and we note that
the outlier from the Raven-M sensitivity example, blackscholes,
actually performs better than the Midway core on Raven-S. Overall we
see similar improvements to the general core, getting a 14\%
improvement over Midway (thanks in part to the benefits seen by the
memory-intensive apps, and a 16\% improvement over its closest
competitor, Tag-team.

\subsection{Workload Comparisons}

Throughout the design of the Raven processors, there have been clearly
different cores generated depending on the the target workload
provided. Even in the case where the SPEC benchmarks matched closely
with the general workload, differences could easily occur based on
what applications were selected and how they were clustered. A stark
difference was seen between the Raven-M and Raven-G processors, and
their performance difference is seen more clearly in figure
\ref{fig:performance}. This is primarily an artifact of optimizing
solely for performance/Watt. Future refinements to \blackBox{}'s
optimization function would easily allow us to incorporate notions of
minimum performance requirements, or to optimize for alternative (e.g. performance$^2$/Watt) metrics. For the stated goal of energy-efficiency, the selected processors perform with reasonable tradeoffs.

 \dcfigure[lib/figures/raven-perf.pdf,{\figtitle{Varying Performance:}
     Between various types of processors, there are clear variations
     in actual performance. This
     variation can change depending on the workload and the designed
     cores. Since \Ravan{} optimizes solely for performance/Watt, it does not always ensure a strong minimum performance bound.},fig:performance,lib/figures/raven-workload.pdf,{\figtitle{Final
       Workload Statistics:} Workload selection plays an important
     role in designing heterogeneous cores, and while there may be
     subtle variations in total workload performance when factoring
     new applications in, the trend shows that Raven processors remain efficient even for applications they neither targeted nor were trained on.}, fig:workload-final]

There is a consistent advantage to using a Raven processor over any of
the other options when considering the metric of performance/watt.
Even with Raven-M where it performed worse than Midway, the difference
was not unacceptable, and it is possible that through a different
initial training set the performance equations can better handle
workloads where all the applications share similar
characteristics. Amongst the Raven-S and Raven-G processor competitors
Raven did better both before and after the introduction of new
applications, and this was done on a relatively small subset of the
available architecture components.  This shows both the promise of the
model and the desire to push forward for discovering new heterogeneous
designs that can capture both energy efficiency and general purpose
applicability.




\section{Conclusion}
\label{sec:conclusion}

In this work, we have presented a model, \blackBox{}, that allows for
heterogeneous core design without the need to exhaustively search for
good cores for a given set of applications. This is done by creating a
series of performance, area, and power equations that predict the
performance of a set of potential heterogeneous cores. These are then
validated for the user via gem5 simulations and McPAT power analysis
to get more accurate values, and these results are then fed back into
the regression training set to allow \blackBox{} to continue learning
about new applications and what works for them. When such processors
are actually tested, we generally see an efficiency improvement of
between 5-10\% over a homogeneous core that was designed with
some energy savings in mind and a basic two-way heterogeneous core
that reflects current heterogeneous design approaches.




%\section*{Acknowledgements}


\bstctlcite{bstctl:etal, bstctl:nodash, bstctl:simpurl}
\bibliographystyle{IEEEtranS}
\bibliography{references}


\end{document}

