\section{Introduction}
\label{sec:introduction}
%\subsection{Multi-Core Problems}
%Discuss multi-core architectures and the problems with power that have come up% as a result of evolving tech

\IEEEPARstart{T}{he} demise of Dennardian scaling~\cite{Dennard74-JSSC-MOSFET_Scaling}
has lead processor designers to increasingly shift their focus from
raw, per-core performance toward energy efficiency as a primary
metric. This transition has given rise to a multitude of multicore
designs across many domains. Compared to their uniprocessor
predecessors, the more modestly aggressive cores in modern multicore
processors reap efficiency benefits not only from avoiding less
rewarding regions of the superlinear relationship between peak core
power and peak core performance, but also from the fact that there are
many applications (e.g. SPECINT-like desktop programs) which will only
rarely approach peak core utilization.

Recent proposals~\cite{Lukefahr12-MICRO-CompositeCores,variable2011multi} and products~\cite{ARM11-WhitePaper-BigLittle} aim to
further capitalize on the latter of these effects by increasing
processor heterogeneity. On a heterogeneous platform, each application
can run on the least energy-expensive processor that still
allows it to achieve a high fraction of its potential performance,
increasing overall efficiency. However, the space of all potential
heterogeneous processors is quite large, even when restricting the
options to traditional core types. It is not yet clear how best to
select the constituent cores in a heterogeneous CMP.
There are two prominent schools of thought for heterogeneous design
strategies, but much room remains in the middle for new
approaches. Current designs, such as ARM's
Big.LITTLE~\cite{ARM11-WhitePaper-BigLittle}, draw inspiration from
early work~\cite{Kumar03-SIHM,Kumar06-PACT-SIHM,Kumar04-SIHM} on
heterogeneity that advocates limiting design costs by combining
previously developed architectures. This, however, may limit the energy
savings of the system, because it maps applications to cores whose
resource diversity owes more to temporal changes in design
restrictions than to a concerted effort to more efficiently serve
particular classes of current or future applications. At the other
extreme, proposals such
as~\cite{Clark05-ISCA-CustomISA,Clark08-ISCA-VEAL,Goulding11-IEEEMICRO-GreenDroid,Venkatesh10-ASPLOS-CCores}
call for domain or even application-level specialization of cores,
which maximizes energy efficiency, but vastly increases design effort
and may produce systems that are overly sensitive to changing
workloads. Taking insight from both ends of the design spectrum, the
intuition is that there will be profitable ground in the middle:
heterogeneous systems that consist of general purpose cores with
traditional components, but where each core in the collection can
optimize its performance/energy tradeoffs for a subset of
applications.

In this paper, we present \blackBox{}, a high-level design approach
for automatically selecting the degree and dimensions of heterogeneity
in a CMP based on the architecture-independent features of a target
workload. \blackBox{} constructs processor configurations out of a
modest sized library of fixed components, but with many possible
combinations. Lifting the restriction that every core be well-suited
for executing the entire workload allows \blackBox{} to select
individual cores with non-standard configurations, e.g. cores lacking
a floating point unit and large caches due to the requirements of part
of the workload, while still effecting general application support through the union of
cores selected in a particular design. With \blackBox{} we can quickly
predict what set of cores would be needed in a single CMP to exploit
heterogeneity within a workload and to produce multiple CMP designs to
better exploit heterogeneity among workloads from different domains.

At a high level, \blackBox{} works as follows: \blackBox{} considers a
design space spanning 8 dimensions including cache
size, issue width, and ALU allocation. We train on a small number
of applications across a subset of the configurations within that
design space. For each training application or application phase, we
collect a vector of architecturally independent features
gathered from MICA~\cite{Hoste07-IEEEMICRO-MICA} and the associated
energy-performance-area three-tuple on a given core. To design a
particular heterogeneous multiprocessor, we pass the architecturally
independent feature vectors of a target workload into a trained
\blackBox{}, as well as the maximum allowed degree of heterogeneity,
\textit{K}. \blackBox{} then partitions the input applications into up
to K groups and, for each group, estimates the energy and performance
of the group across all points in the design space. Finally,
\blackBox{} picks the optimal performance/watt configuration across
each group, selects that processor for inclusion in the CMP, and
provides a distance-based mapping function between the
architecture-independent features of any current or future application
and the predicted-preferred core type.

The contributions of this work include the following:
\begin{itemize}
\item We describe \blackBox{}, a design approach for automatically
  selecting an energy-efficient set of heterogeneous cores, given an
  architecturally independent description of a workload.
\item We show that \blackBox{} designs with greater degrees of
  heterogeneity offer superior energy efficiency compared to
  homogeneous or limited heterogeneity designs, improving on existing
  designs by up to 18\%.
\item We show that there is sufficient variation among workloads, and that
  different workloads warrant different heterogeneous solutions.
\item We explore the sensitivity of \blackBox{} designs to workload
  variation and show that \Ravan{} performs well against other potential
  solutions.
\end{itemize}


% LocalWords:  Dennardian multicore uniprocessor superlinear SPECINT
% LocalWords:  CMP ARM's tradeoffs ALU tuple
